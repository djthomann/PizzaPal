\makeglossaries

\newglossaryentry{template}{%
  name={Template},
  description={Eine Vorlage für eine Packung mit Maßen, Gewicht, Tragkraft und einer Zutat als Inhalt sowie einer Liste von Unverträglichkeiten}
}

\newglossaryentry{zutat}{%
  name={Zutat},
  description={Eine Zutat in einer Pizzeria}
}

\newglossaryentry{paket}{%
  name={Paket},
  description={Eine konkrete Instanz eines Template}
}

\newglossaryentry{unvertraeglichkeit}{%
  name={Unverträglichkeit},
  description={Eine Zutat mit der ein Paket nicht im selben Regalabschnitt stehen darf}
}

\newglossaryentry{lager}{%
  name={Lager},
  description={Eine Anordnung von Regalen}
}

\newglossaryentry{brett}{%
  name={Brett},
  plural={Bretter},
  description={Ein horizontales Brett in einem Regal}
}

\newglossaryentry{stuetze}{%
  name={Stütze},
  plural={Stützen},
  description={Ein vertikales Brett, die Begrenzung eines Regalteils}
}

\newglossaryentry{regal}{%
  name={Regal},
  description={Alle Stützen und Bretter zusammengefasst}
}

\newglossaryentry{regalteil}{%
  name={Regalteil},
  description={Der Teil eines Regals zwischen zwei Stützen}
}

\newglossaryentry{regalabschnitt}{%
  name={Regalabschnitt},
  description={Ein durch zwei benachbarte Bretter und zwei benachbarte Stützen abgeteilter Bereich eines Regals}
}

\newglossaryentry{stapel}{%
  name={Stapel},
  description={Eine Sammlung von Paketen, die aufeinander im Regal stehen. Zusammen bewegbar}
}
