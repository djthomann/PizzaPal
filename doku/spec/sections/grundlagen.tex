\section{Projektgrundlagen}
\subsection{Einführung \& Zielstellung}
\begingroup
Es soll ein System zur einfachen Verwaltung eines \textbf{\gls{lager}s} von \textbf{\gls{zutat}en}
einer Pizzeria ersellt werden. Ein Lager ist vereinfacht dargestellt, als zweidimensionaler
Raum, in dem ein Regal mit Paketen von Zutaten steht. 
Ein \textbf{\gls{paket}} muss über eine Tragkraft, Maße, ein Gewicht, eine Menge der Zutat
und eine Liste von Unverträglichkeiten (Zutaten, in deren Nähe die verpackte Zutat schlecht würde) verfügen. 
Die Pakete werden in einer 
grafischen Oberflächen vereinfacht als Rechtecke in einer 
Regalstruktur (ebenfalls aufgebaut aus Rechtecken) verwaltet. Ein \textbf{\gls{regal}} besteht
dabei aus \textbf{\glspl{stuetze}} und \textbf{\glspl{brett}n}, die vom Nutzer per Drag\&Drop angepasst bzw. umgebaut werden kann
und so ergeben sich \textbf{\gls{regalabschnitt}e} und \textbf{\gls{regalteil}e}, siehe Abbildung~\ref{fig:Regalmodell} auf Seite~\pageref{fig:Regalmodell}.
Pakete sollen zunächst als Vorlagen (\textbf{\gls{template}s}) angelegt werden, die dann dem Regal
per Drag\&Drop hinzugefügt werden. Die Erstellungs- und Bearbeitunsschritte sollen in einer separaten Eingabemaske geschehen.
Es soll möglich sein, neue Pakete dem 
Lager hinzuzufügen und diese auch wieder zu löschen (dem Lager zu entnehmen), oder gegebenenfalls anzupassen. Per Drag\&Drop sollen 
Die Pakete sollen neu angeordnet werden können, ohne, dass dabei ein Brett überlastet wird oder
ein Paket zusammen mit einer \textbf{\gls{unvertraeglichkeit}} (unvertäglichen Zutat) gelagert wird.
Pakete sollen auf Bretter gestapelt werden können (\textbf{\gls{stapel}} mehrere Pakete auf einem anderen, aber nicht über die Breite des unteren Pakets hinaus) und dann auch als Stapel verschoben werden können. 
Probleme beim Umlagern oder Erstellen von Paketen, sowie bei der Regalkonfiguration sollen dem Nutzer grafisch angezeigt werden.
\endgroup
\subsection{Rahmenbedingungen \& Technische Anforderungen}
\begingroup 
Das Programm benötigt mindestens Java 21 und 8 GB Arbeitsspeicher um zu funktionieren. Für die 
Grafische Benutzeroberfläche ist eine Bildschirm mit mindestens Full-HD-Auflösung und
RGB-Farbausgabe. Zur Interaktion mit dem Programm ist eine Maus erforderlich. Die einwandfreie Funktion des
Programms wird nur unter einer aktuellen (Kubuntu 22.04.4 LTS) Linux-Version
garantiert. Für die Speicherung des Lagers als Datei ist Speicherplatz von mindestens
10 GB zur Verfügung zu stellen.