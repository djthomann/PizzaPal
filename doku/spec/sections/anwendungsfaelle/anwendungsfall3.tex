\subsubsection*{}
\textbf{Regal anlegen}\bigskip\\
\textbf{Autor}: David Thomann\\
\textbf{Akteure}: Chef, Aushilfskraft\\
\textbf{Fachlicher Auslöser:} Ein neues Regal wurde aufgebaut und soll jetzt im virtuellen Lager erscheinen\\
\textbf{Vorbedingungen: }Lager ist noch nicht voll belegt\\
\textbf{Standardablauf:}
\begin{enumerate}
    \item System: Geht in den Regal-Erstellungs-Modus
    \item Akteur: Erstellt Stützen- und Brettvorlagen, sofern sie noch nicht existieren
    \item Akteur: Zieht Stützen und Bretter aus der Vorlagenliste per Drag\&Drop und ordnet sie entsprechend des echten Regals ans
    \item Akteur: Eingabe bestätigen
    \item System: Legt Regal an
    \item System: Verlässt Regal-Erstellungs-Modus
\end{enumerate}
\textbf{Alternative Abläufe}
\begin{enumerate}
    \item[] \begin{enumerate}\setcounter{enumi}{8}
\item System stellt fest, dass es bereits ein Paket mit dieser Bezeichnung gibt und lehnt Erstellung ab
    \begin{enumerate}
    \item System schlägt vor, eine neue Vorlage zu erstellen oder Menge eines bereits existierenden Pakets anzupassen
    \item weiter bei 3
\end{enumerate}
\end{enumerate}
\end{enumerate}
\textbf{Fehlersituationen:}
\begin{enumerate}
    \item Benutzer wählt Tragkraft für ein Brett, die unter dem Gewicht der aktuellen darauf stehenden Pakete liegt
    \item Benutzer versucht Regal abzuspeichern ohne Stützen am Anfang/Ende des Regals
\end{enumerate}
\textbf{Sonderfälle:}
\begin{enumerate}
\item System lehnt neue Maße für Stützen und Bretter ab, da sie keinen Sinn ergeben (negative Werte, Maße, die nicht ins Lager passen). Der Fehler wird grafisch angezeigt.
\item Akteur korrigiert betroffene Eingaben
\item weiter bei 6
\end{enumerate}
\textbf{Nachbedingung/Ergebnis:}\\
Das System ist wieder im ursprünglichen Modus, das Regal ist mit allen vorher beinhalteten Zutaten in einem neuen Zustand\\
\textbf{Nichtfunktionale Anforderungen:}\\
Plausibitätsfehler werden vom System in max. 200ms angezeigt und können direkt korrigiert werden.\\
\textbf{Parametrisierbarkeit/Flexibilität:}\\
-\\
\textbf{Nutzungshäufigkeit/Mengengerüst:}\\
1-2 mal pro Jahr