\subsubsection*{}
\textbf{Paketvorlage bearbeiten}\bigskip\\
\textbf{Autor}: Nha Dan-Tran\\
 \textbf{Akteure}: Chef, Aushilfskraft\\
 \textbf{Fachlicher Auslöser:} Änderungen der Paketgewichte/größe\\
 \textbf{Vorbedingungen: }Pakete sind bereits in der Datenbank angelegt\\
 \textbf{Standardablauf:} 

\begin{enumerate}
    \item Chef/Aushilfskraft wählt zu bearbeitende Vorlage aus dem Inventar aus
    \item Chef/Aushilfskraft ändert entsprechende Felder
    \item System überprüft Eingaben auf Plausibilität
    \item System lässt Eingaben vom Nutzer nbestätigen
    \item Chef/Aushilfskraft bestätigt Eingabe
    \item System aktualisiert die Eingaben in der Datenbank und die Vorlage
\end{enumerate}

\textbf{Alternative Abläufe/ Fehlersituationen / Sonderfälle:}
\begin{enumerate}\setcounter{enumi}{3}
    \item[] \begin{enumerate}
        \item System akzeptiert Eingabe nicht
        \begin{enumerate}
            \item System hebt nicht plausible Eingaben farbig hervor und bittet den Nutzer selbige anzupassen
            \item Nutzer passt Eingaben an
            \item System überprüft Eingaben erneut auf Plausibilität
            \item System akzeptiert Eingaben und lässt den Nutzer bestätigen
            \item Nutzer bestätigt
            \item System aktualisiert Datenbank und Vorlage
        \end{enumerate}
    \end{enumerate}
\end{enumerate}

\noindent\textbf{Nachbedingung/Ergebnis:}\\
Inventar wurde aktualisiert und Pakete sind zur weiteren Verteilung im Regalsystem bereit.

\noindent\textbf{Nichtfunktionale Anforderungen:\\}
Reaktionszeit < 5. sek.

\noindent\textbf{Parametrisierbarkeit/Flexibilität:}\\
Jedes Paket lässt sich somit individuell anpassen.

\noindent\textbf{Nutzungshäufigkeit/Mengengerüst:}\\
Ca. 1 mal im Monat.