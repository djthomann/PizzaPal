\section{Randbedingungen}
\subsection{Technische Randbedingung}

\begin{longtable}{|m{0.3\textwidth}|m{0.7\textwidth}|}
    \hline
    \textbf{Randbedingungen} & \textbf{Erläuterungen bzw. Hintergrund}\\
    \hline 
    Hardwareaustattung & Zur Interaktion mit dem Programm ist eine Tastatur und eine Maus erforderlich.\\
    \hline    
    Betrieb auf Linux Desktop Betriebssystemen & Die Software richtet sich speziell an kleine Pizzerien, die oft weniger IT-Ressourcen haben und daher von einem stabilen, sicheren und kostenfreien Betriebssystem profitieren.\\
    \hline
    Implementierung in Java & Es wird mindestens Java 21 benötigt.\\
    \hline
    GUI Implementierung mit JavaFX & JavaFX ermöglicht die Entwicklung plattformunabhängiger Anwendungen, die auf verschiedenen Betriebssystemen ohne größere Anpassungen laufen können. Dies ist besonders nützlich für zukünftige Erweiterungen auf andere Betriebssysteme.\\
    \hline
\end{longtable}

\subsection{Organisatorische Randbedingungen}
\begin{longtable}{|m{0.3\textwidth}|m{0.7\textwidth}|}
    \hline
    \textbf{Randbedingungen} & \textbf{Erläuterungen bzw. Hintergrund}\\
    \hline 
    Team & David Thomann, Aron Schlegel, Nha-Dan Tran\\
    \hline
    Vorgehensmodell & Iterativ. Die Dokumentation erfolgt unter Einsatz des arc42 Templates.\\
    \hline
    Entwicklungswerkzeuge & Draw.io und PlantUML für Diagramme. Latex für die Erzeugung des Anforderungs- und Architekturdokuments. Visual Studio Code zur Erstellung der Quelltexte. Die Software muss aber auch allein mit Gradle, also ohne IDE erzeugbar sein.\\
    \hline
    Konfigurations- und Versionsverwaltung & Rhodecode\\
    \hline
    Testwerkzeuge und -prozesse & (geplant) JUnit im Annotationsstil für Integrationtstests.\\
    \hline
    Veröffentlichung & Geplant als Open Source auf GitHub zur Verfügung zur stellen. Lizenz: GNU General Public License version 3.0 (GPLv3). \\
    \hline
\end{longtable}